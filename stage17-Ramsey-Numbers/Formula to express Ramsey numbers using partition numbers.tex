\documentclass[12pt]{article}
\usepackage{graphicx}
\usepackage{enumerate}
\usepackage[ruled]{algorithm2e}
\usepackage{amsmath}
\usepackage{hyperref}
\usepackage{listings}

\input amssym.def 
\input amssym.tex 

\bibliographystyle{plain}

\renewcommand{\arraystretch}{1.2}

%%%%%%%
\setlength{\textwidth}{6.3in}
\setlength{\textheight}{8.7in}
\setlength{\topmargin}{0pt}
\setlength{\headsep}{0pt}
\setlength{\headheight}{0pt}
\setlength{\oddsidemargin}{0pt}
\setlength{\evensidemargin}{0pt}
%%%%%%%

\newtheorem{theorem}{Theorem}[section]
%\newtheorem{theorem}[theorem]{Theorem}
\newtheorem{proposition}[theorem]{Proposition}
\newtheorem{lemma}[theorem]{Lemma}
\newtheorem{axiom}[theorem]{Axiom}
\newtheorem{corollary}[theorem]{Corollary}
\newtheorem{conjecture}[theorem]{Conjecture}
\newtheorem{heuristic}[theorem]{Heuristic}
\newtheorem{principle}[theorem]{Principle}
\newtheorem{question}[theorem]{Question}
\newtheorem{claim}[theorem]{Claim}

\begin{document}

\title{Formula to express Ramsey numbers $R(m,n)$ using partition numbers}

\author{
JiangLin Luo \\
\small WangYueHu Community 1-pian 7-dong \\[-0.8ex]
\small Changsha, China 230026 \\
\small\tt cody@ustc.edu \\
}

\maketitle

\begin{center}
\small{Mathematics Subject Classifications: 05C15, 05C55, 05C85}
\end{center}

\begin{abstract}
This paper proposes Fence Conjecture: Ramsey Number 
\[R(m,n)=(2m-1)*p(2m-6+n,m)+\{1,m,m+1\}, \text{for } 3\leq m\leq n\]. 
Here $p(n,k)$ denotes the number of partitions of $n$ into exactly $k$ parts. The last summand chooses one from three values $\{1,m,m+1\}$.
When $m=3 \text{ or }4$, it is 
\[R(3,n)=5*p(n,3)+\{1,3,4\}, \text{for } 3<=n\] 
\[R(4,n)=7*p(n+2,4)+\{1,4,5\}, \text{for } 4<=n\] .
\end{abstract}

\section{Introduction}

The classical two color Ramsey numbers $R(r, s)$ in Ramsey's theorem is the minimum number of vertices, 
$v = R(m, n)$, such that all undirected simple graphs of order v, contain a clique of order m, or an independent set of order n.
Or to say in any $2$-coloring of the edges of the complete graph $K_v$, there
is a monochromatic copy of $K_m$ in color $1$ or of $K_n$ in color $2$.
By symmetry, $R(m, n) = R(n, m)$.
Radziszowski's survey on Small Ramsey Numbers \cite{survey} gives the basic terminology and
the current status of a host of problems related to Ramsey numbers. 

\begin{table}[]
  \caption{divmod(R(3,n)-1,5)}
  \label{tab:divmodR3n1}
  \begin{tabular}{lllllllllll}
  n             & 1 & 2 & 3 & 4 & 5 & 6 & 7 & 8 & 9 & 10 \\
  (R(3,n)-1)/5  & 0 & 0 & 1 & 1 & 2 & 3 & 4 & 5 & 7 & ?8 \\
  (R(3,n)-1)\%5 & 0 & 2 & 0 & 3 & 3 & 2 & 2 & 2 & 0 & ?0
  \end{tabular}
\end{table}

Brendan McKay listed some Ramsey Graphs \cite{RamseyGraphs}. 
Around 2010, I observed that $R(3,n)-1\equiv 0,2,3 (\text{mod } 5)$ for $n=1..9$ those all known $R(3,n)$.
This observation is the basis of the following conjecture.

\section{R(m,n) Formula Conjecture}\label{FormulaConjecture}

\begin{conjecture}[Fence Conjecture part1]\label{FCpart1}
  When constructing critical Ramsey graph for $R(3,n)$, 
  the $R(3,n)-1$ vertices can be laid out interlacing with 2 vertices and 3 vertices. 
  This means  \[R(3,n)-1 \equiv 0,2,3 (\text{mod }5)\] . 
\end{conjecture}

\begin{table}[]
  \caption{layout Ramsey critical graphs for $R(3,n)-1$ by interlacing with 2 or 3 vertices}
  \label{tab:interlacing23}
  \begin{tabular}{lllllll}
  o & o & o & o & o & o & o \\
    & o &   & o &   & o &   \\
  o & o & o & o & o & o & o
  \end{tabular}
  \end{table}

  
In 2023-11-22, When I searched $0,1,1,2,3,4,5,7,8$ in OEIS, I got A001399 \cite{A001399} partitions function $p(n,3)$.
So I proposed, 

\begin{conjecture}[Fence Conjecture part2]\label{FCpart2}
  \[R(3,n)=5*p(n,3)+\{1,3,4\}, \text{for } n\geq 3\] .
\end{conjecture}

Also, the other two known Ramsey numbers have $R(4,4)=18=7*2+4; R(4,5)=25=7*3+4$. 
It appears that $R(m, n)$ formula has modular form and relates to partitions function closely.

\begin{conjecture}[Fence Conjecture]\label{FC}
Ramsey numbers
\[R(m,n)=(2m-1)*p(2m-6+n,m)+\{1,m,m+1\}, \text{for } 3\leq m\leq n\].
\end{conjecture}

Here partitions function $p(n,k)$ \cite{A008284} or denotes as $p_k(n)$, is both the number of partitions of $n$ into exactly $k$ parts, and the number of partitions of $n$ into parts of maximum size exactly $k$. 
These two types of partition conjugate in their Young diagrams. 
The last summand chooses one value among three integers $\{1,m,m+1\}$.


\section{Experiments and Data}

\begin{lstlisting}[language=Python]
def p(n,k):
  return Partitions(n,length=k).cardinality()

P=matrix(ZZ,30,30,lambda n,k:p(n,k))

top=15
data=[[(2*m - 1)*p(2*m - 6 + n, m) for n in (m..top)] for m in (3..top)]
print(data)
\end{lstlisting}

\begin{table}[]
  \caption{(2m-1)*p(2m-6+n,m)}
  \label{tab:FCdata15}
\begin{center}
\begin{tabular}{||c c c c c c c c c c c c c c||}
  \text{} & n=3 & 4 & 5 & 6 & 7 & 8 & 9 & 10 & 11 & 12 & 13 & 14 & 15 \\
  \hline
  m=3 & 5 & 5 & 10 & 15 & 20 & 25 & 35 & 40 & 50 & 60 & 70 & 80 & 95 \\
  4 & \text{} & 14 & 21 & 35 & 42 & 63 & 77 & 105 & 126 & 161 & 189 & 238 & 273 \\
  5 & \text{} & \text{} & 45 & 63 & 90 & 117 & 162 & 207 & 270 & 333 & 423 & 513 & 630 \\
  6 & \text{} & \text{} & \text{} & 121 & 154 & 220 & 286 & 385 & 484 & 638 & 781 & 990 & 1210 \\
  7 & \text{} & \text{} & \text{} & \text{}  & 273 & 364 & 494 & 637 & 845 & 1066 & 1365 & 1703 & 2132  \\
  8 & \text{} & \text{} & \text{} & \text{} & \text{} & 600 & 780 & 1050 & 1335 & 1740 & 2190 & 2790 & 3450 \\
  9 & \text{} & \text{} & \text{} & \text{} & \text{} & \text{} & 1241 & 1598 & 2091 & 2669 & 3417 & 4284 & 5406 \\
  10 & \text{} & \text{} & \text{} & \text{} & \text{} & \text{} & \text{} & 2432 & 3116 & 4028 & 5073 & 6460 & 8037 \\
  11 & \text{} & \text{} & \text{} & \text{} & \text{} & \text{} & \text{} & \text{} & 4599 & 5838 & 7455 & 9345 & 11760  \\
  12 & \text{} & \text{} & \text{} & \text{} & \text{} & \text{} & \text{} & \text{} & \text{} & 8418 & 10580 & 13386 & 16675 \\
  13 & \text{} & \text{} & \text{} & \text{} & \text{} & \text{} & \text{} & \text{} & \text{} & \text{} & 14925 & 18675 & 23375 \\
  14 & \text{} & \text{} & \text{} & \text{} & \text{} & \text{} & \text{} & \text{} & \text{} & \text{} & \text{} & 25839 & 32076 \\
  15 & \text{} & \text{} & \text{} & \text{} & \text{} & \text{} & \text{} & \text{} & \text{} & \text{} & \text{} & \text{} & 43732 \\
\end{tabular}
\end{center}
\end{table}

Ramsey numbers have basic inequality relation $R(r,s)\leq R(r-1,s)+R(r,s-1)$, 
the conjectured formula satisfies this inequality completely. And it implies that $R(3,10)=41, R(5,5)=46, R(6,6)=122 \text{ or } 127 \text{ or } 128$.
The last summand has not be decided from ${1,m,m+1}$ yet.   
Fence conjecture needs to be supplemented or even corrected.  


\bibliographystyle{plain}
\begin{thebibliography}{99}

\bibitem{survey}
S. P. Radziszowski,
\newblock Small Ramsey Numbers,
\newblock {\em The Electronic Journal of Combinatorics}, DS1, 2014

\bibitem{A008284}
OEIS Foundation Inc.~(2021), The On-Line Encyclopedia of Integer Sequences, \url{https://oeis.org/A008284}.

\bibitem{A001399}
OEIS Foundation Inc.~(2021), The On-Line Encyclopedia of Integer Sequences, \url{https://oeis.org/A001399}.

\bibitem{RamseyGraphs}
Brendan McKay,
\newblock Ramsey Graphs, \url{http://users.cecs.anu.edu.au/~bdm/data/ramsey.html}.

\end{thebibliography}

\end{document}
